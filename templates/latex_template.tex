\documentclass[10pt]{article}         %% What type of document you're writing.
%\documentclass[aps,prb,reprint,longbibliography]{revtex4-1}
\usepackage[letterpaper,bindingoffset=0.0in,left=1in,right=1in,top=1in,bottom=1in,footskip=.25in]{geometry}

\usepackage{amssymb}
\usepackage{color}
\usepackage{amsmath}
\usepackage{grffile}
\usepackage{cancel}
\usepackage{hyperref}
%\usepackage{calrsfs} % for \cal

% The following metadata will show up in the PDF properties
\hypersetup{
  pdfproducer= none,  % producer of the document
  pdfcreator = none,  % producer of the document
  colorlinks = true,
  allcolors = black, %-- use this if you want to set all links to the same color
}

%\usepackage{comment}
%\excludecomment{figure}
%\documentclass[aps,prb,reprint,groupedaddress]{revtex4-1}
%\linespread{1.5}
%\documentclass[aps,rmp,reprint,groupedaddress]{revtex4-1}
%\documentclass[aps,rmp,reprint,groupedaddress]{revtex4-1}
\usepackage{amssymb}
\usepackage{color}
\usepackage{amsmath}

\usepackage{array}

\usepackage{bm}
\usepackage[english]{babel}
% You should use BibTeX and apsrev.bst for references
% Choosing a journal automatically selects the correct APS
% BibTeX style file (bst file), so only uncomment the line
% below if necessary.
%\bibliographystyle{apsrev4-1}

% include graphics with the command \includegraphics
\usepackage{graphicx}
\usepackage{tikz}
\usetikzlibrary{positioning}
\usepackage{ifthen}
\usepackage{xstring}

\usepackage[T1]{fontenc}

%\newcommand{\fn}{ubiquitin_stretch}
%\newcommand{\fn}{nanosheet_birth}
%\newcommand{\fn}{nanosheet_traj}
\newcommand{\fn}{FILEBASE}
\newcommand{\centerit}[1]{\noindent\textcolor{white}{.}\hfil#1\hfil\newline}

\date{\vspace{-5ex}}

\begin{document}
%\title{The ${\mathcal{R}}$eport for \url{\fn}}

%\begin{abstract}
%SOMETHING
%\end{abstract}

%\maketitle must follow title, authors, abstract, \pacs, and \keywords
%\maketitle
\centerit{{\huge The ${\mathcal{R}}$eport for \texttt{\url{\fn}.pdb}}}

\vfill

\begin{tabular}{m{0.3\linewidth} m{0.3\linewidth} m{0.3\linewidth}}
\hfil \LARGE Structure \hfil & \hfil \LARGE Ramachandran \hfil & \hfil \LARGE\textcolor{white}{.}~~ ${\mathcal{R}}$ code \hfil \\
\hfil                        & \hfil \LARGE plot \hfil         &                                         \\
\hfil\includegraphics[width=0.2\textwidth]{\fn/\fn_pdb_first.pdb.dat.png}\hfil&
\hfil\includegraphics[height=0.3\textwidth]{\fn/\fn_ramachandran_first.eps}\hfil&
\hfil\includegraphics[height=0.25\textwidth]{\fn/\fn_bars_first.eps}\hfil\\
\end{tabular}

\vfill

\centerit{{\LARGE${\mathcal{R}}$ code}}
\newline
\centerit{\includegraphics[height=0.28\textwidth]{\fn/\fn_bars.eps}}
\centerit{Each column represents an ${\mathcal{R}}$-histogram (${\mathcal{R}}$ code)}
\centerit{of one frame within \texttt{\url{\fn}.pdb}}

\vfill
\centerit{{\LARGE${\mathcal{R}}$ graph}}
\newline
\centerit{\includegraphics[height=0.3\textwidth]{\fn/\fn_resno_vs_r_model_first.eps}}
\centerit{Relationship between residue \# and $\mathcal{R}$ (first frame)}

\vfill

\end{document}